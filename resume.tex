\documentclass[10pt]{article}
\parindent 0pt

\usepackage{mathpazo}
\usepackage{etaremune}
\usepackage{nopageno}
\usepackage{amsmath,amstext,color}
\usepackage[margin=0.75in]{geometry}
\usepackage{enumitem}
\usepackage{parskip}
\usepackage{fancyhdr}
\usepackage[hidelinks]{hyperref}

\fancyhead[C]{Gillen Brown --- Resume}
\renewcommand{\headrulewidth}{0pt}
\fancyfoot{}
\pagestyle{fancy}

\newcommand{\todo}[1]{\textcolor{red}{[\bf #1]}}

% command for spacer in contact info
% \newcommand{\spacer}{\enspace \rule[-4pt]{1pt}{15pt} \enspace}
\newcommand{\spacer}{$\bullet$ \ }

\newcommand{\header}[1]{\vspace{1.5em}\par \textbf{\large #1}\strut\hrule\vspace{-0.9em}}
%par makes it always go to the next line, strut produces slight vertical offset
% for the rule.

\newcommand{\actionHeader}[2]{\vspace{0.6em}\textbf{#1} \hfill #2}
\newcommand{\actionHeaderSecondLine}[2]{\newline \textit{#1} \hfill #2}
\newcommand{\indentedItemDate}[2]{\newline\null\qquad #1 \hfill #2}
\newcommand{\indentedItem}[1]{\newline\null\qquad #1}
\newcommand{\itemWithName}[2]{\null {\bf #1}: #2\newline}
% null lets the indents work after the space. Newlines allows me to have less
% whitespace in the Latex file.

\title{Gillen Brown}

\begin{document}
\thispagestyle{empty}

\setlist[itemize]{noitemsep, topsep=-0.2em}

\begin{center}
{\LARGE \bf Gillen Brown}

{\normalsize \href{mailto:gillenbrown@gmail.com}{gillenbrown@gmail.com} \spacer \href{https://www.gillenbrown.com}{gillenbrown.com} \spacer \href{http://www.linkedin.com/in/gillenbrown}{linkedin.com/in/gillenbrown}}
\end{center}

% Hiding the summary for now, to save space
% \bigskip
% Future Ph.D graduate with extensive technical and analytic skills, seeking position in data science or computer science. \todo{redo to emphasize how I've used many skills in PhD work and looking to transition}

% \medskip
\vspace{1em}
\header{Education}
\actionHeader{University of Michigan}{Ann Arbor, MI}
\indentedItemDate{Ph.D. Astronomy and Astrophysics}{April 2022}
% \indentedItem{{\bf Topic:} Chemical Evolution of Simulated Milky Way Sized Galaxies}
% \indentedItem{{\bf Summary:} Added functionality to existing C codebase that models galaxy formation, then ran }
% \indentedItem{simulations on supercomputers such as NASA's Pleiades cluster to understand how different elements}
% \indentedItem{are distributed within galaxies.}
\indentedItem{GPA: 4.0/4.0}
\vspace{-0.5em}

\actionHeader{University of Missouri-Kansas City}{Kansas City, MO}
\indentedItemDate{B.S. Physics with emphasis in Astronomy}{May 2016}
\indentedItem{Minors in Mathematics and Computer Science}
\indentedItem{GPA: 4.0/4.0  (Summa Cum Laude)}
\vspace{-0.3em}

\header{Research Experience}
\actionHeader{University of Michigan}{Ann Arbor, MI}
\actionHeaderSecondLine{Graduate Student Research Assistant}{September 2016 --- April 2022}
\begin{itemize}
    \item Performed astrophysics research resulting in four peer-reviewed publications
    \item Used Python to analyze astrophysical data of several kinds, including tables, images, and volumetric data
    \item Created a Bayesian model to measure the radii of $\sim$6000 star clusters seen in Hubble Space Telescope images 
    \item Statistically characterized the population of star cluster radii to test theories of star cluster evolution
    % (keep if more computation focused) \item Updated existing C codebase that computationally models galaxy formation to improve the modeling of star cluster formation and evolution
    % \item Developed a method of customizing the initial conditions for galaxy formation simulations to improve computational performance 
    \item Developed a method to improve the computational efficiency of numerical simulations of galaxy formation
    \item Analyzed the properties of star clusters in numerical simulations of galaxy formation
    % \item Awarded more than 15 million CPU-hours on supercomputers including Stampede2, Anvil, and Frontera
    \item Presented work at local seminars and large scientific conferences (audiences from 5 to 100+ people)
\end{itemize}

\actionHeader{University of Missouri-Kansas City}{Kansas City, MO}
\actionHeaderSecondLine{Undergraduate Researcher}{June 2014 -- May 2016} 
\begin{itemize}
    \item Used Python to analyze astrophysical images and tabular data
    \item Developed an automated method to estimate the redshift of galaxy clusters
    % \item Created tutorials teaching basic coding and research methods to new group members
\end{itemize}

\header{Leadership and Teamwork Experience}
\actionHeader{Michgan Dark Skies}{Ann Arbor, MI}
\actionHeaderSecondLine{Co-coordinator}{September 2019 --- April 2022}
\begin{itemize}
    \item Coordinated the activities of this group with 100+ members working to prevent light pollution
    \item Worked with Central Student Government to pass a resolution encouraging U of M to reduce light pollution
\end{itemize}

% \actionHeader{University of Michigan}{Ann Arbor, MI}
% \actionHeaderSecondLine{Graduate Student Instructor}{September 2017 --- April 2021}
% \begin{itemize}
%     \item Facilitated lab sections for four introductory level astronomy courses ($\sim$100 students per course)
%     % \item Assisted in developing course materials for introductory level astronomy courses 
%     \item Sole instructor for 50 student course ``Naked Eye Astronomy''
%     \item Mentored 10 other graduate student instructors 
% \end{itemize}

% \actionHeader{University of Michigan}{Ann Arbor, MI}
% \actionHeaderSecondLine{Graduate Student Representative}{September 2018 --- April 2021}
% \begin{itemize}
%     \item Chosen by other astronomy graduate students as our representative for several important department committees, including the curriculum committee and a committee reforming the Ph.D. qualifying exam
% \end{itemize}

\actionHeader{University of Michigan Museum of Natural History}{Ann Arbor, MI}
\actionHeaderSecondLine{Science Communication Fellow}{May 2019 -- March 2020}
\begin{itemize}
    \item Created hands-on activity for children demonstrating the amounts of different elements in the Universe
    \item Delivered multiple presentations on astronomy for museum visitors of all ages
\end{itemize}

\header{Selected Awards}
\vspace{1.1em}
\begin{itemize}
    \item {\bf 2020 Michigan Institute of Data Science: Basketball Data Madness Challenge} --- Won a UM competition by extracting actionable insights from basketball player performance data
    \item {\bf 2021 Astronomy DEI Champion} --- Awarded for helping form an antiracism reading group
\end{itemize}


\header{Skills}
\vspace{0.2em}
\itemWithName{Technical Skills}{Python (including SciPy, NumPy, etc.), C/C++, Unix, LaTeX, git}
\itemWithName{Data Analysis}{Statistics and probability, linear regression, Bayesian modeling, data visualization}
\itemWithName{Soft Skills}{Technical writing, public speaking, critical thinking, problem solving}

\end{document}